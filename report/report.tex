%@descr: wzór sprawozdania, raportu lub pracy - nadaje się do przeróbek
%@author: Maciej Komosiński

\documentclass{article} 
\usepackage{polski} 
\usepackage[utf8]{inputenc} 
\usepackage[OT4]{fontenc} 
\usepackage{graphicx,color} %include pdf's (and png's for raster graphics... avoid raster graphics!) 
\usepackage{url} 
\usepackage[pdftex,hyperfootnotes=false,pdfborder={0 0 0}]{hyperref} %za wszystkimi pakietami; pdfborder nie wszedzie tak samo zaimplementowane bo specyfikacja nieprecyzyjna; pod miktex'em po prostu nie widac wtedy ramek


% Zmiana rozmiarów strony tekstu
\addtolength{\voffset}{-1cm}
\addtolength{\hoffset}{-1cm}
\addtolength{\textwidth}{2cm}
\addtolength{\textheight}{2cm}

%bardziej zyciowe parametry sterujace rozmieszczeniem rysunkow
\renewcommand{\topfraction}{.85}
\renewcommand{\bottomfraction}{.7}
\renewcommand{\textfraction}{.15}
\renewcommand{\floatpagefraction}{.66}
\renewcommand{\dbltopfraction}{.66}
\renewcommand{\dblfloatpagefraction}{.66}
\setcounter{topnumber}{9}
\setcounter{bottomnumber}{9}
\setcounter{totalnumber}{20}
\setcounter{dbltopnumber}{9}

% własny bullet list z malymi odstepami
\newenvironment{tightlist}{
\begin{itemize}
  \setlength{\itemsep}{1pt}
  \setlength{\parskip}{0pt}
  \setlength{\parsep}{0pt}}
{\end{itemize}}

%obrazkow szukamy w nastepujacym katalogu:
\graphicspath{{pics/}}



%\title{Sprawozdanie z laboratorium:\\Metaheurystyki i Obliczenia Inspirowane Biologicznie}
%\author{}
%\date{}


\begin{document}

\thispagestyle{empty} %bez numeru strony

\begin{center}
{\large{Sprawozdanie z laboratorium:\\
Metaheurystyki i obliczenia inspirowane biologicznie}}

\vspace{3ex}

Część I: Algorytmy optymalizacji lokalnej, problem ATSP
%Część II: Algorytmy optymalizacji lokalnej i globalnej, problem QAP
%Część III: Eksperyment: ... (prezentację można zrobić w LaTeX - służy do tego klasa "beamer")

\vspace{3ex}
{\footnotesize\today}

\end{center}


\vspace{10ex}

Prowadzący: dr hab.~inż. Maciej Komosiński

\vspace{5ex}

Autorzy:
\begin{tabular}{lllr}
    \textbf{Marcin Mrugas} & inf122580 & marcin.mrugas@student.put.poznan.pl \\
    \textbf{Piotr Kicki} & inf122401 & piotr.kicki@student.put.poznan.pl \\
\end{tabular}

\vspace{5ex}

Zajęcia środowe, 16:50.

\vspace{35ex}

\noindent Oświadczam/y, że niniejsze sprawozdanie zostało przygotowane wyłącznie przez powyższych autora/ów,
a wszystkie elementy pochodzące z innych źródeł zostały odpowiednio zaznaczone i~są cytowane w bibliografii.  

\newpage


\section*{Udział autorów}
\begin{tightlist}
\item MM zaimplementował szkielet aplikacji, oraz przeszukiwanie losowe, przeprowadził eksperyment..., opisał..., przygotował...
\item PK zaimplementował przeszukiwanie zachłanne..., przeprowadziła eksperyment..., opisała..., przygotowała...
\end{tightlist}






\section{Wstęp}

%\begin{figure} %obrazek pojawia się przed pierwszym odwołaniem do niego -- to przydatna zasada
%\begin{center}
%\includegraphics[width=0.4\textwidth]{rys_graf.pdf}
%\end{center}
%\caption{Przykładowy schemat z programu \emph{graphviz} -- narzędzia do automatycznego generowania schematów~\cite{graphviz}. Przerywane strzałki oznaczają, że wszędzie gdzie się da używamy grafiki wektorowej -- unikamy wstawiania bitmap do dokumentu. W niektórych przypadkach użycie bitmap jest uzasadnione (w celu szybkiego podglądu na ekranie lub dla niezwykle skomplikowanych grafik, zawierających np.~setki tysięcy obiektów). Różnice w grafice rastrowej i wektorowej omawia prezentacja~\url{https://www.youtube.com/watch?v=_98SDNIpm24}.}
%\label{fig:schemat}
%\end{figure}


To jest przykładowy tekst w LaTeX. Przeczytaj go uważnie (treść, jego źródło oraz \%komentarze) i użyj tego źródła \texttt{*.tex} jako szablonu sprawozdania -- to źródło pokazuje jak

\begin{tightlist}
\item wstawić schemat stworzony graphviz'em (Rys.~\ref{fig:schemat}),
\item wstawić wykres stworzony gnuplot'em (Rys.~\ref{fig:1Tdelta}, \ref{fig:3d}, \ref{fig:every} i \ref{fig:regr}) oraz matplotlib'em (Rys.~\ref{fig:matplotlib}),
\item zacytować literaturę sformatowaną przez bibtex~\cite{MiOIB,Goldberg-2002},
\item odwoływać się do rysunków, cytowań i części sprawozdania (np.\ rozdział~\ref{sec:typografia}).
\end{tightlist}







\section{Cechy dobrego sprawozdania}

Dobre sprawozdanie
\begin{tightlist}
\item pozwala odtworzyć samodzielnie czytelnikowi eksperyment (od danych po wyniki),
\item nie zawiera niedomówień,
\item przedstawia wnioski uporządkowane od ogólnych do szczegółowych,
\item cytuje literaturę w tekście,
\item nie zawiera zbyt obszernych listingów,
\item czytelnie prezentuje wyniki -- zwykle za pomocą wykresów,
\item wszelkie dane liczbowe pokazuje z właściwą liczbą miejsc znaczących,
\item jest zwięzłe i estetyczne.
\end{tightlist}

\subsection{Typografia}
\label{sec:typografia}

Pamiętajmy o różnicy pomiędzy łącznikiem\footnote{Przeczytaj w Wikipedii opis hasła ,,Dywiz''.} a myślnikiem -- a także o cytowaniu wszelkich materiałów źródłowych w odpowiednich miejscach~\cite{WikiDash}. Cytujmy konkretną stronę, a nie ogólny adres witryny. Cudzysłowy polskie piszemy metodą ,,przecinków i apostrofów''.

Do sprawdzania pisowni bezpośrednio w pliku\ .tex służy między innymi program \emph{aspell}. Rozumie on różne sposoby kodowania polskich literek, a także ma wbudowane filtry do html'a i innych popularnych formatów. Dzięki tym filtrom pomija słowa kluczowe typowe dla danego formatu pliku, analizując tylko właściwy tekst. %backslash przed kropką i spacją podpowiada LaTeX'owi, żeby użył zwykłej spacji (a nie poszerzonej), ponieważ kropka za spacją nie jest końcem zdania (LaTeX domyślnie robi większe przerwy przed wszystkimi kropkami zakładając, że kropki oddzielają zdania).







\section{Wykresy}

\noindent Do przetwarzania tekstowych plików z wynikami oraz rysowania wykresów wyśmienicie nadaje się python wzbogacony o bibliotekę matplotlib. Zdecydowanie warto się ich nauczyć! Jeśli jednak chciał(a)byś wykorzystać program gnuplot (co jest mniej przyszłościowe), to jego nowe wersje posiadają już niezły terminal `pdf', więc można zrezygnować z pośrednictwa formatu ps/eps. Jeśli nie wiesz jak zrobić jakiś rodzaj wykresu, sprawdź stronę z przykładami gnuplota~\cite{GnuplotDemos}.

Zanim przygotujesz wykres, obejrzyj koniecznie porady dotyczące ich tworzenia -- jak zrobić czytelny i profesjonalny wykres:~\url{https://www.youtube.com/watch?v=pfSgcsQ2Mtk}. %tylda to "nierozłączna spacja" -- skleja sąsiadujące wyrazy.


\begin{figure}
\begin{center}
\includegraphics[width=0.8\textwidth]{rys_wykres2d.pdf}
\end{center}
\caption{Przykładowy wykres z gnuplota, terminal postscript, zamieniony na pdf za pomocą programu epstopdf z dystrybucji LaTeX'a (czasem eps2pdf). Wykres pokazuje różnice $\Delta_{dir}$ wartości $p_{dir}$ dla kąta $90^\circ$.}
\label{fig:1Tdelta}
\end{figure}

\begin{figure}
\begin{center}
\includegraphics[width=0.9\textwidth]{rys_wykres3d.pdf}
\end{center}
\caption{Jeszcze jeden przykładowy wykres z gnuplota.}
\label{fig:3d}
\end{figure}

\begin{figure}
\begin{center}
\includegraphics[width=0.85\textwidth]{rys_gnuplot_every.pdf}
\end{center}
\caption{Przykład filtrowania danych do wykresu (\emph{every}) oraz użycie własnych funkcji i formuł w gnuplocie.}
\label{fig:every}
\end{figure}

\begin{figure}
\begin{center}
\includegraphics[width=0.85\textwidth]{rys_gnuplot_regr.pdf}
\end{center}
\caption{Przykład regresji: gnuplot ma wbudowany moduł dopasowujący do danych parametry funkcji o dowolnej zadanej postaci. Pozwala też definiować makra, prowadzić obliczenia i umieszczać na wykresie etykiety.}
\label{fig:regr}
\end{figure}

\begin{figure}
\begin{center}
\includegraphics[width=0.48\textwidth]{rys_short_scalar_2_02.png}\hfill\includegraphics[width=0.48\textwidth]{rys_short_scalar_2_02-kulki.png}
\end{center}
\caption{Przykład wizualizacji w python+matplotlib; tutaj dane 5D pokazane na dwa sposoby w 3D. Wstawione bitmapy (nieprawidłowo, powinna być postać wektorowa), a wykresy są tu za małe (nieczytelne).}
\label{fig:matplotlib}
\end{figure}

\clearpage %pozwol LaTeX'owi umiescic zaległe rysunki od razu tutaj -- "uwalnia" nagromadzone zaległości, dzięki temu nie wylądują na końcu dokumentu





%%%%%%%%%%%%%%%% literatura %%%%%%%%%%%%%%%%

\bibliography{sprawozd}
\bibliographystyle{plainurl}


\end{document}
