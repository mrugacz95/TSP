%@descr: wzór sprawozdania, raportu lub pracy - nadaje się do przeróbek
%@author: Maciej Komosiński

\documentclass{article} 
\usepackage{polski} 
\usepackage[utf8]{inputenc} 
\usepackage[OT4]{fontenc} 
\usepackage{graphicx,color} %include pdf's (and png's for raster graphics... avoid raster graphics!) 
\usepackage{url} 
\usepackage[pdftex,hyperfootnotes=false,pdfborder={0 0 0}]{hyperref} %za wszystkimi pakietami; pdfborder nie wszedzie tak samo zaimplementowane bo specyfikacja nieprecyzyjna; pod miktex'em po prostu nie widac wtedy ramek


\input{_settings.tex}

%\title{Sprawozdanie z laboratorium:\\Metaheurystyki i Obliczenia Inspirowane Biologicznie}
%\author{}
%\date{}


\begin{document}

\thispagestyle{empty} %bez numeru strony

\begin{center}
{\large{Sprawozdanie z laboratorium:\\
Metaheurystyki i obliczenia inspirowane biologicznie}}

\vspace{3ex}

Część I: Algorytmy optymalizacji lokalnej, problem ATSP
%Część II: Algorytmy optymalizacji lokalnej i globalnej, problem QAP
%Część III: Eksperyment: ... (prezentację można zrobić w LaTeX - służy do tego klasa "beamer")

\vspace{3ex}
{\footnotesize\today}

\end{center}


\vspace{10ex}

Prowadzący: dr hab.~inż. Maciej Komosiński

\vspace{5ex}

Autorzy:
\begin{tabular}{lllr}
    \textbf{Marcin Mrugas} & inf122580 & marcin.mrugas@student.put.poznan.pl \\
    \textbf{Piotr Kicki} & inf122401 & piotr.kicki@student.put.poznan.pl \\
\end{tabular}

\vspace{5ex}

Zajęcia środowe, 16:50.

\vspace{35ex}

\noindent Oświadczam/y, że niniejsze sprawozdanie zostało przygotowane wyłącznie przez powyższych autora/ów,
a wszystkie elementy pochodzące z innych źródeł zostały odpowiednio zaznaczone i~są cytowane w bibliografii.  

\newpage


\section*{Udział autorów}
\begin{tightlist}
\item MM zaimplementował szkielet aplikacji, oraz przeszukiwanie losowe, przeprowadził eksperyment..., opisał..., przygotował...
\item PK zaimplementował przeszukiwanie zachłanne..., przeprowadziła eksperyment..., opisała..., przygotowała...
\end{tightlist}






\section{Wstęp}

%\begin{figure} %obrazek pojawia się przed pierwszym odwołaniem do niego -- to przydatna zasada
%\begin{center}
%\includegraphics[width=0.4\textwidth]{rys_graf.pdf}
%\end{center}
%\caption{Przykładowy schemat z programu \emph{graphviz} -- narzędzia do automatycznego generowania schematów~\cite{graphviz}. Przerywane strzałki oznaczają, że wszędzie gdzie się da używamy grafiki wektorowej -- unikamy wstawiania bitmap do dokumentu. W niektórych przypadkach użycie bitmap jest uzasadnione (w celu szybkiego podglądu na ekranie lub dla niezwykle skomplikowanych grafik, zawierających np.~setki tysięcy obiektów). Różnice w grafice rastrowej i wektorowej omawia prezentacja~\url{https://www.youtube.com/watch?v=_98SDNIpm24}.}
%\label{fig:schemat}
%\end{figure}


Naszym zadanie było zaimplementowanie i przestudiowanie asymetrycznego problemu komiwojażera. Problem ten polega na znalezieniu najkrótszego cyklu Hamiltona w grafie asymetrycznym. Można go interpretować jako logistyczny problem dotyczący znalezienia takiej trasy dla dostawcy aby odwiedził wszystkie miasta w jak najkrótszym czasie. Problem należy do klasy problemów NP-zupełnych. Złożoność tego problemu, w najgorszym przypadku, sprowadza się do przeglądnięcia wszystkich cyklów Hamiltona których dla grafu o zupełnego o n wierzchołkach jest n!. Dlatego problem ten optymalizuje się algorytmami które w przestrzeni rozwiązań starają się rozważać mniejszą część rozwiązań i uzyskują rozwiązanie gorsze od optymalnego ale osiągają wynik w znacznie krótszym czasie.

\begin{figure} 
\begin{center}
\includegraphics[width=0.7\textwidth]{TSP_USA.png}
\end{center}
\caption{Przykładowe rozwiązanie problemu komiwojażera polegającego na znalezieniu trasy pomiędzy stolicami stanów Stanów Zjednoczonych.}
\label{fig:schemat}
\end{figure}


\section{Opis sąsiedztwa}

Operatorem sąsiedztwa w naszej implementacji jest 2-OPT. Nowy sąsiad jest tworzony poprzez zamianę miejscami dwóch wierzchołków rozwiązania. 

\begin{figure} 
\begin{center}
\includegraphics[width=0.4\textwidth]{twooptwiki.pdf}
\end{center}
\caption{Przykład sąsiedztwa.}
\label{fig:schemat}
\end{figure}


\section{Porównanie działania algorytmów}


\subsection{Jakość}
\subsection{Czas działania}
\subsection{Efektywność}
\subsection{Średnia liczba kroków algorytmów}


\section{Jakość rozwiązania początkowego, a jakość rozwiązania końcowego}
\section{Jakość rozwiązania w zależności od ilości uruchomień}
\section{Podobieństwo rozwiązań}
\section{Wnioski}

\section{Trudności i problemy}

\section{Propozycje udoskonaleń}

\clearpage

\bibliography{report}
\bibliographystyle{plainurl}


\end{document}
